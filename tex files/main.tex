\documentclass{report}
\usepackage[utf8]{inputenc}
\usepackage{enumitem}
\title{Assignment 9 (GATE, EC 2017-16)}
\author{Manasa}
\date{December 2020}

\usepackage{circuitikz}


\begin{document}
\maketitle
\section{QUESTION}
\begin{figure}[htp]
        \centering
        \includegraphics[width=17cm]{Figure.pdf}
         \caption{}
        \label{fig:ok}
\end{figure}
\huge
Consider the circuit shown in the figure.The Boolean expression F implemented by the circuit is

 \begin{enumerate}
 \item X'Y'Z' + XY + Y'Z
\item X'YZ' + XZ + Y'Z
\item X'YZ' + XY + Y'Z
\item X'Y'Z' + XZ + Y'Z
  \end{enumerate}



\section{SOLUTION}
\subsection{MUX:}

A multiplexer (sometimes spelled multiplexor and also known as a MUX) is defined as a combinational circuit that selects one of several data inputs and forwards it to the output. The inputs to a multiplexer can be analog or digital. Multiplexers are also known as data selectors.

\subsection{Why is it required:}

In digital systems, many times it is necessary to select a single data line from several data-input lines and the data from the selected data input line should be available on the output line. The digital circuit which does this task is a multiplexer.

\subsection{How does a multiplexer work:}

The multiplexer works like a multiple-input and single-output switch. The output gets connected to only one of the n data inputs at a given instant of time. Therefore, the multiplexer is ‘many into one’ and it works as the digital equivalent of an analog selector switch.

\subsection{Answer}
From figure 1,
\newline 
In first multiplexer the input signals are Y and 0,control line is X.According to this the output signal is X'Y
\newline
In second multiplexer the input signals are X'Y and (X'Y)' control line is Z.The output signal is F.
\newline
\newline
F = X'YZ' + ((X'Y)')Z
\newline
\newline
  = X'YZ' + (X + Y')Z   \hspace{3}       (Using demorgan laws)
\newline\newline
= X'YZ' + XZ + Y'Z 
\newline
So the boolean expression F implemented by the circuit is
\newline
\begin{equation}
   \textcolor{blue}{F = X'YZ'+XZ+Y'Z}
\end{equation}
 

\newpage
\section{CIRCUIT}
\begin{figure}[htp]
        \centering
        \includegraphics[width=17cm]{Detailed figure.pdf}
         \caption{}
        \label{fig:ok}
\end{figure}
\newline
From the above figure we can derive the expression of F
\newline
\begin{equation}
   \textcolor{blue}{F = X'YZ'+XZ+Y'Z}
\end{equation}


\newpage
\section{TRUTH TABLE of Expression F}


\begin{table}[]\centering
    \begin{tabular}{|l|l|l|l|}
         \hline
        X & Y & Z & F  \\ \hline
        0 & 0 & 0     & 0  \\
        0 & 0 & 1     & 1  \\
        0 & 1 & 0     & 1   \\ 
        0 & 1 & 1     & 0   \\
        1 & 0 & 0     & 0   \\
        1 & 0 & 1     & 1   \\
        1 & 1 & 0     & 0   \\
        1 & 1 & 1     & 1    \\ 
        \hline

\end{tabular}
\end{table}


\end{document}